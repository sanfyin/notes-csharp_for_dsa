\section{String Handling in C\#}

In C\#, a string is a built in and primitive data type. The primitive data type string maps to the System.String class. The objects of the String
class (or string) are immutable by nature. By immutable it means that the state of the object can not be changed by
any operation. This is the reason why when we call the ToUpper() method on string, it doesn’t change the original
string but creates and returns a new string object that is the upper case representation of the original object. The
mutable version of string in the .Net Platform is System.Text.StringBuilder class. The objects of this class are
mutable, that is, their state can be changed by their operations. Hence, if you call the Append() method on a
StringBuilder class object, the new string will be appended to (added to the end of) the original object. Let’s now
discuss the two classes one by one.\\

\textbf{The string class and its members}\\

We have been using the string class since our first lesson in the C\# school. We have also seen some of its properties
(like Length) and methods (like Equals()) in previous lessons. Here, we will describe some of the common
properties and methods of the String class and then demonstrate their use in code.

\begin{center}
    \begin{tabular}{ | m{5em} | m{10cm} | } 
    \hline
    Property or Method & Description \\
    \hline
    Length & Gets the number of characters the String object contains.\\
    CompareTo(string s) & Compares this instance of the string with the supplied string s and returns an
    integer on the pattern of the IComparable interface.\\
    Compare(string s1, string s2) & This is a static method and compares the two supplied strings on the pattern of
    the IComparable interface.\\
    Equals(string s) & Returns true if the supplied string is exactly the same as this string, else returns
    false.\\
    Concat(string s) & Returns a new string that is the concatenation (addition) of this and the supplied
    string s.\\
    Insert(int index, string s) & Returns a new string by inserting the supplied string s at the specified index of
    this string.\\
    Copy(string s) & This static method returns a new string that is the copy of the supplied string.\\
    Intern(string s) & This static method returns the system’s reference to the supplied string.\\
    StartsWith(string s) & Returns true if this string starts with the supplied string s.\\
    EndsWith(string s) & Returns true if this string ends with the supplied string s.\\
    IndexOf(string s)

    IndexOf(char ch)&
    Returns the zero based index of the first occurrence of the supplied string s or
    supplied character ch. This method is overloaded and more versions are
    available.\\
    LastIndexOf(string s)

    LastIndexOf(char ch)&
    Returns the zero based index of the last occurrence of the supplied string s or
    supplied character ch. This method is overloaded and more versions are
    available.\\
    Replace(char, char)

    Replace(string, string) &
    Returns a new string by replacing all occurrences of the first char with the
    second char (or first string with the second string).\\
    Split(params char[]) & Identifies those substrings (in this instance) which are delimited by one or more
    characters specified in an array, then places the substrings into a String array and
    returns it.\\
    Substring(int i1)

    Substring(int i2, int i3) &
    Retrieves a substring of this string starting from the index position i1 till the end
    in the first overloaded form. In the second overloaded form, it retrieves the
    substring starting from the index i2 and which has a length of i3 characters.\\
    ToCharArray() & Returns an array of characters of this string.\\
    ToUpper() & Returns a copy of this string in uppercase.\\
    ToLower() & Returns a copy of this string in lowercase.\\
    Trim() & Returns a new string by removing all occurrences of a set of specified characters
    from the beginning and at end of this instance.\\
    TrimStart() & Returns a new string by removing all occurrences of a set of specified characters
    from the beginning of this instance.\\
    TrimEnd() & Returns a new string by removing all occurrences of a set of specified characters
    from the end of this instance.\\
    \hline
    \end{tabular}
\end{center}


In the following program we have demonstrated the use of most of the above methods of the string class. The
program is quite self-explanatory and only includes the method calls and their results.

\begin{lstlisting}
    using System;
    namespace CSharpSchool
    {
        class Test
        {
            static void Main()
            {
                string s1 = "sanfy";
                string s2 = "sfy";
                string s3 = "sanfyin";
                string s4 = "C# is a great programming language!";
                string s5 = " This is the target text ";
                Console.WriteLine("Length of {0} is {1}", s1, s1.Length);
                Console.WriteLine("Comparision result for {0} with {1} is {2}", s1, s2, s1.CompareTo(s2));
                Console.WriteLine("Equality checking of {0} with {1} returns {2}", s1, s3, s1.Equals(s3));
                Console.WriteLine("Equality checking of {0} with lowercase {1} ({2}) returns {3}",
                s1, s3, s3.ToLower(), s1.Equals(s3.ToLower()));
                Console.WriteLine("The index of a in {0} is {1}", s3, s3.IndexOf('a'));
                Console.WriteLine("The last index of a in {0} is {1}", s3, s3.LastIndexOf('a'));
                Console.WriteLine("The individual words of `{0}' are", s4);
                string []words = s4.Split(' ');
                foreach(string word in words)
                {
                    Console.WriteLine("\t {0}", word);
                }
                Console.WriteLine("\nThe substring of \n\t`{0}' \nfrom index 3 of length 10 is \n\t`{1}'",
                s4, s4.Substring(3, 10));
                
                Console.WriteLine("\nThe string \n\t`{0}'\nafter trimming is \n\t`{1}'", s5, s5.Trim());
            }
        }
    }    
\end{lstlisting}

The output of the program will certainly help you understand the behavior of each member.


\begin{lstlisting}
    Length of sanfy is 5
    Comparison result for sanfy with snfy is -1
    Equality checking of sanfy with Sanfy returns False
    Equality checking of faraz with lowercase Faraz (faraz) returns True
    The index of a in Faraz is 1
    The last index of a in Faraz is 3
    The individual words of `C# is a great programming language!' are
    C#
    is
    a
    great
    programming
    language!
    The substring of
    `C# is a great programming language!'
    from index 3 of length 10 is
    `is a great'
    The string
    ` This is the target text '
    after trimming is
    `This is the target text'       
\end{lstlisting}

\textbf{The StringBuilder class}\\

The System.Text.StringBuilder class is very similar to the System.String class with the difference that it is
mutable; that is, the internal state of its objects can be modified by its operations. Unlike in the string class, you
must first call the constructor of a StringBuilder to instantiate its object.

\begin{lstlisting}
    string s = "This is held by string";
    StringBuilder sb = new StringBuilder("This is held by StringBuilder");        
\end{lstlisting}


StringBuilder is somewhat similar to ArrayList and other collections in the way that it grows automatically as the
size of the string it contains changes. Hence, the Capacity of a StringBuilder may be different from its Length.
Some of the more common properties and methods of the StringBuilder class are listed in the following table:

\begin{center}
    \begin{tabular}{ | m{10em} | m{10cm} | } 
    \hline
    Property or Method & Description \\
    \hline
    Length & Gets the number of characters that the StringBuilder object contains.\\
    Capacity & Gets the current capacity of the StringBuilder object.\\
    Append() & Appends the string representation of the specified object at the end of this
    StringBuilder instance. The method has a number of overloaded forms.\\
    Insert() & Inserts the string representation of the specified object at the specified index of
    this StringBuilder object.\\
    Replace(char, char)

    Replace(string, string) & 
    Replaces all occurrences of the first supplied character (or string) with the
    second supplied character (or string) in this StringBuilder object.\\
    Remove(int st, int length) & Removes all characters from the index position st of specified length in the
    current StringBuilder object.\\
    Equals(StringBuilder) & Checks the supplied StringBuilder object with this instance and returns true if
    both are identical; otherwise, it returns false.\\
    \hline
    \end{tabular}
\end{center}

The following program demonstrates the use of some of these methods:

\begin{lstlisting}
    using System;
    using System.Text;
    namespace CSharpSchool
    {
        class Test
        {
            static void Main()
            {
                StringBuilder sb = new StringBuilder("The text");
                string s = " is complete";
                Console.WriteLine("Length of StringBuilder `{0}' is {1}", sb, sb.Length);
                Console.WriteLine("Capacity of StringBuilder `{0}' is {1}", sb, sb.Capacity);
                Console.WriteLine("\nStringBuilder before appending is `{0}'", sb);
                Console.WriteLine("StringBuilder after appending `{0}' is `{1}'", s, sb.Append(s));
                Console.WriteLine("\nStringBuilder after inserting `now' is `{0}'", sb.Insert(11, "
                now"));
                Console.WriteLine("\nStringBuilder after removing 'is ' is `{0}'", sb.Remove(8, 3));
                Console.WriteLine("\nStringBuilder after replacing all `e' with `x' is {0}",
                sb.Replace('e', 'x'));
            }
        }
    }
        
\end{lstlisting}


And the output of the program is:

\begin{lstlisting}
    Length of StringBuilder `The text' is 8
    Capacity of StringBuilder `The text' is 16
    StringBuilder before appending is `The text'
    StringBuilder after appending ` is complete' is `The text is complete'
    StringBuilder after inserting `now' is `The text is now complete'
    StringBuilder after removing 'is ' is `The text now complete'
    StringBuilder after replacing all `e' with `x' is Thx txxt now complxtx        
\end{lstlisting}

Note that in all cases, the original object of the StringBuilder is getting modified, hence StringBuilder objects are
mutable compared to the immutable String objects.
