\section{Flow Control And Conditional Statements}

\textbf{The if...else statement}\\
Condition checking has always been the most important construct in any language right from the time of the
assembly language days. C\# provides conditional statements in the form of the if...else statement. The structure of
this statement is:

if(Boolean expression)\\
Statement or block of statements\\
else\\
Statement or block of statements\\


The else clause above is optional. A typical example is:

\begin{lstlisting}
    if(i==5)
    {
        Console.WriteLine("Thank God, I finally became 5.");
    }
\end{lstlisting}

In the above example, the console message will be printed only if the expression i==5 evaluates to true. If you
would like to take some action when the condition does not evaluate to true, then you can use else clause:

\begin{lstlisting}
    if(i==5)
    {
        Console.WriteLine ("Thank God, I finally became 5.");
    }
    else
    {
        Console.WriteLine("Missed...When will I become 5?");
    }
\end{lstlisting}

Only the first message will be printed if i is equal to 5. In any other case (when i is not 5), the second message will
be printed. \\

You can also have if after else for further conditioning:

\begin{lstlisting}
    if(i==5) // line 1
    {
        Console.WriteLine("Thank God, I finally became 5.");
    }
    else if(i==6) // line 5
    {
        Console.WriteLine("Ok, 6 is close to 5.");
    }
    else // line 9
    {
        Console.WriteLine("Missed...When will I become 5 or be close to 5?");
    }
\end{lstlisting}


Here else if(i==6) is executed only if the first condition i==5 is false, and else at line 9 will be executed only if the
second condition i==6 (line 5) executes and fails (that is, both the first and second conditions fail). The point here
is else at line 9 is related to if on line 5.\\

Since if...else is also a statement, you can use it under other if...else statement (nesting), like:

\begin{lstlisting}
    if(i>5) // line 1
    {
        if(i==6) // line 3
        {
            Console.WriteLine("Ok, 6 is close to 5.");
        }
        else // line 7
        {
        Console.WriteLine("Oops! I'm older than 5 but not 6!");
        }
        Console.WriteLine("Thank God, I finally became older than 5.");
    }
    else // line 13
    {
        Console.WriteLine("Missed...When will I become 5 or close to 5?");
    }
\end{lstlisting}

The else on line 7 is clearly related to if on line 3 while else on line 13 belongs to if on line 1. Finally, do note
(C/C++ programmers especially) that if statements expect only Boolean expressions and not integer values. It’s an
error to write:

\begin{lstlisting}
    int flag = 0;
    if(flag = 1)
    {
        // do something...
    }
\end{lstlisting}

Instead, you can either use:

\begin{lstlisting}
    int flag = 0;
    if(flag == 1) // note ==
    {
        // do something...
    }
\end{lstlisting}

or,

\begin{lstlisting}
    bool flag = false;
    if(flag = true) // Boolean expression
    {
        // do something...
    }
\end{lstlisting}

The keys to avoiding confusion in the use of any complex combination of if...else are aligning the code to enhance readability. If you are using Visual Studio or some other
editor that supports the language, the editor will do indentation for you. Otherwise, you have to take care of this
yourself.\\

\textbf{The switch...case statement}\\

If you need to perform a series of specific checks, switch...case is present in C\# just for this. The general structure
of the switch...case statement is:

\begin{lstlisting}
    switch(integral or string expression)
    {
        case constant-expression:
        statements
        breaking or jump statement
        // some other case blocks
        ...
        default:
        statements
        breaking or jump statement
    }    
\end{lstlisting}

It takes less time to use switch...case than using several if...else if statements. Let’s look at it with an example:

\begin{lstlisting}
    using System;
    class SwitchCaseExample
    {
        // Demonstrates the use of switch...case statement along with
        // the use of command line argument
        static void Main(string [] userInput)
        {
            int input = int.Parse(userInput[0]);
            // convert the string input to integer.
            // Will throw a run-time exception if there is no input at run-time or if
            // the input is not castable to integer.
            switch(input) // what is input?
            {
                case 1: // if it is 1
                Console.WriteLine("You typed 1 (one) as the first command line argument");
                break; // get out of switch block
                case 2: // if it is 2
                Console.WriteLine("You typed 2 (two) as the first command line argument");
                break; // get out of switch block
                case 3: // if it is 3
                Console.WriteLine("You typed 3 (three) as the first command line argument");
                break; // get out of switch block
                default: // if it is not any of the above
                Console.WriteLine("You typed a number other than 1, 2 and 3");
                break; // get out of switch block
            }
        }
    }    
\end{lstlisting}

The program must be supplied with an integer command line argument. First, compile the program (at the
command line or in Visual Studio.Net). Suppose we made an exe with name "SwitchCaseExample.exe", we would
run it at the command line like this:

Let’s get to internal working. First, we converted the first command line argument (userInput[0]) into an int
variable input. For conversion, we used the static Parse() method of the int data type. This method takes a string
and returns the equivalent integer or raises an exception if it can’t. Next we checked the value of input variable
using a switch statement:

\begin{lstlisting}   
    switch(input)
    {
    ...
    }    
\end{lstlisting}

Later, on the basis of the value of input, we took specific actions under respective case statements. Once our case
specific statements end, we mark it with the break statement before the start of another case (or the default) block.

\begin{lstlisting}
    case 3: // if it is 3
    Console.WriteLine("You typed 3 (three) as first command line argument");
    break; // get out of switch block        
\end{lstlisting}

If all the specific checks fail (input is none of 1,2 and 3), the statements under default executes.

\begin{lstlisting}
    default:
    // if it is not any of the above
    Console.WriteLine ("You typed a number other than 1, 2 and 3");
    break; // get out of switch block        
\end{lstlisting}

There are some important points to remember when using the switch...case statement in C\#:

\begin{itemize}
    \item You can use either integers (enumeration) or strings in a switch statement
    \item The expression following case must be constant. It is illegal to use a variable after case:
    case i: // incorrect, syntax error
    \item You can use multiple statements under single case and default statements:
    \item A colon : is used after the case statement and not a semicolon ;
    case "India":
    continent = "Asia";
    Console.WriteLine("India is an Asian Country");
    break;
    default:
    continent = "Un-recognized";
    Console.WriteLine
    ("Un-recognized country discovered");
    break;
    \item The end of the case and default statements is marked with break (or goto) statement. We don’t use {}
    brackets to mark the block in switch...case as we usually do in C\#
    \item C\# does not allow fall-through. So, you can’t leave case or default without break statement (as you can in
    Java or C/C++). The compiler will detect and complain about the use of fall-through in the switch...case
    statement.
    \item The break statement transfers the execution control out of the current block.
    \item Statements under default will be executed if and only if all the case checks fail.
    \item It is not necessary to place default at the end of switch...case statement. You can even place the default
    block before the first case or in between cases; default will work the same regardless of its position.
    However, making default the last block is conventional and highly recommended. Of course, you can’t
    have more than one default block in a single switch...case.
    
\end{itemize}


\textbf{Loops In C\#}\\
Loops are used for iteration purposes, i.e., doing a task multiple times (usually until a termination condition is met)

\textbf{The for Loop}\\
The most common type of loop in C\# is the for loop. The basic structure of a for loop is exactly the same as in Java
and C/C++ and is:

\begin{lstlisting}
    for(assignment; condition; increment/decrement)
        statements or block of statements
        enclosed in {} brackets
        
\end{lstlisting}

Let's see a for loop that will write the integers from 1 to 10 to the console:


\begin{lstlisting}
    for(int i=1; i<=10; i++)
    {
        Console.WriteLine("In the loop, the value of i is {0}.", i);
    }    
\end{lstlisting}

At the start, the integer variable i is initialized with the value of 1. The statements in the for loop are executed while
the condition (i<=10) remains true. The value of i is incremented (i++) by 1 each time the loop starts.\\

Some important points about the for loop\\

All three statements in for(), assignment, condition and increment/decrement are optional. You can use any
combination of these and even decide not to use any one at all. This would make what is called and ’Indefinite or
infinite loop’ (a loop that will never end until or unless the break instruction is encountered inside the body of the
loop). But, you still have to supply the appropriate semi colons:

\begin{lstlisting}
    for(; ;)
    for( ; i<10; i++)
    for(int i=3; ; i--)
    for( ; i>5; )        
\end{lstlisting}

If you don’t use the {} brackets, the statement immediate following for() will be treated as the iteration statement.
The example below is identical to the one given above:

\begin{lstlisting}
    for(int i=1; i<=10; i++)
        Console.WriteLine("In the loop, value of i is {0}.", i);
        
\end{lstlisting}

I will again recommend that you always use {} brackets and proper indentation.
If you declare a variable in for()’s assignment, its life (scope) will only last inside the loop and it will die after the
loop body terminates (unlike some implementations of C++). Hence if you write:

\begin{lstlisting}
    for(int i=1; i<=10; i++)
    {
        Console.WriteLine("In the loop, value of i is {0}.", i);
    }
    i++; // line 1    
\end{lstlisting}

The compiler will complain at line 1 that "i is an undeclared identifier".
You can use break and continue in for loops or any other loop to change the normal execution path. break
terminates the loop and transfers the execution to a point just outside the for loop:

\begin{lstlisting}
    for(int i=1; i<=10; i++)
    {
        if(i>5)
        {
            break;
        }
        Console.WriteLine("In the loop, value of i is {0}.", i);
    }    
\end{lstlisting}

The loop will terminate once the value of i gets greater than 5. If some statements are present after break, break
must be enclosed under some condition, otherwise the lines following break will become unreachable and the
compiler will generate a warning (in Java, it’s a syntax error).

\begin{lstlisting}
    for(int i=3; i<10; i++)
    {
        break; // warning, Console.WriteLine (i); is unreachable code
        Console.WriteLine(i);
    }    
\end{lstlisting}

continue ignores the remaining part of the current iteration and starts the next iteration.

\begin{lstlisting}
    for(int i=1; i<=10; i++)
    {
        if(i==5)
        {
            continue;
        }
        Console.WriteLine("In the loop, value of i is {0}.", i);
    }    
\end{lstlisting}


Console.WriteLine will be executed for each iteration except when the value of i becomes 5. The sample output of
the above code is:

\begin{lstlisting}
    In the loop, value of i is 1.
    In the loop, value of i is 2.
    In the loop, value of i is 3.
    In the loop, value of i is 4.
    In the loop, value of i is 6.
    In the loop, value of i is 7.
    In the loop, value of i is 8.
    In the loop, value of i is 9.        
    In the loop, value of i is 10.
\end{lstlisting}


\textbf{The do...while Loop}\\

The general structure of a do...while loop is

\begin{lstlisting}
    do
        Statement or block of statements
    while(boolean expression);
        
\end{lstlisting}

The statements under do will execute the first time and then the condition is checked. The loop will continue while
the condition remains true.\\

The program for printing the integers 1 to 10 to the console using the do...while loop is:

\begin{lstlisting}
    int i=1;
    do
    {
        Console.WriteLine("In the loop, value of i is {0}.", i);
        i++;
    } while(i<=10);
\end{lstlisting}

Some important points here are:

\begin{itemize}
    \item The statements in a do...while() loop always execute at least once.
    \item There is a semicolon ; after the while statement.
\end{itemize}

\textbf{while Loop}\\

The while loop is similar to the do...while loop, except that it checks the condition before entering the first iteration
(execution of code inside the body of the loop). The general form of a while loop is:

\begin{lstlisting}
    while(Boolean expression)
        statements or block of statements
        
\end{lstlisting}

Our program to print the integers 1 to 10 to the console using while will be:

\begin{lstlisting}
    int i=1;
    while(i<=10)
    {
        Console.WriteLine("In the loop, value of i is {0}.", i);
        i++;
    }        
\end{lstlisting}


