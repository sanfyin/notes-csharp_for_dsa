\section{Arrays in C\#}

\textbf{Array Declaration}\\

An Array is a collection of values of a similar data type. Technically, C\# arrays are a reference type. Each array in
C\# is an object and is inherited from the System.Array class. Arrays are declared as:

\begin{lstlisting}
    <data type> [] <identifier> = new <data type>[<size of array>];    
\end{lstlisting}

Let's define an array of type int to hold 10 integers.

\begin{lstlisting}
    int [] integers = new int[10];    
\end{lstlisting}

The size of an array is fixed and must be defined before using it. However, you can use variables to define the size
of the array:

\begin{lstlisting}
    int size = 10;
    int [] integers = new int[size];        
\end{lstlisting}

You can optionally do declaration and initialization in separate steps:

\begin{lstlisting}
    int [] integers;
    integers = new int[10];        
\end{lstlisting}

It is also possible to define arrays using the values it will hold by enclosing values in curly brackets and separating
individual values with a comma:

\begin{lstlisting}
    int [] integers = {1, 2, 3, 4, 5};    
\end{lstlisting}

This will create an array of size 5, whose successive values will be 1, 2, 3, 4 and 5.\\

\textbf{Accessing the values stored in an array}\\

To access the values in an Array, we use the indexing operator [int index]. We do this by passing an int to indicate
which particular index value we wish to access. It’s important to note that index values in C\# start from 0. So if an
array contains 5 elements, the first element would be at index 0, the second at index 1 and the last (fifth) at index 4.

The following lines demonstrate how to access the 3rd element of an array:

\begin{lstlisting}
    int [] intArray = {5, 10, 15, 20};
    int j = intArray[2];        
\end{lstlisting}

Let’s make a program that uses an integral array.

\begin{lstlisting}
    // demonstrates the use of arrays in C#
    static void Main()
    {
        // declaring and initializing an array of type integer
        int [] integers = {3, 7, 2, 14, 65};
        // iterating through the array and printing each element
        for(int i=0; i<5; i++)
        {
            Console.WriteLine(integers[i]);
        }
    }        
\end{lstlisting}

Here we used the for loop to iterate through the array and the Console.WriteLine() method to print each individual
element of the array. Note how the indexing operator [] is used.

The above program is quite simple and efficient, but we had to hard-code the size of the array in the for loop. As we
mentioned earlier, arrays in C\# are reference type and are a sub-class of the System.Array Class. This class has lot
of useful properties and methods that can be applied to any instance of an array that we define. Properties are very
much like the combination of getter and setter methods in common Object Oriented languages. Properties are
context sensitive, which means that the compiler can un-ambiguously identify whether it should call the getter or
setter in any given context. We will discuss properties in detail in the coming lessons. System.Array has a very
useful read-only property named Length that can be used to find the length, or size, of an array programmatically.\\

Using the Length property, the for loop in the above program can be written as:

\begin{lstlisting}
    for(int i=0; i<integers.Length; i++)
    {
        Console.WriteLine(integers[i]);
    }    
\end{lstlisting}

This version of looping is much more flexible and can be applied to an array of any size and of any data-type.\\

Now we can understand the usual description of Main(). Main is usually declared as:

\begin{lstlisting}
    static void Main(string [] args)    
\end{lstlisting}

The command line arguments that we pass when executing our program are available in our programs through an
array of type string identified by the args string array.\\

\textbf{foreach Loop}\\

There is another type of loop that is very simple and useful to iterate through arrays and collections. This is the
foreach loop. The basic structure of a foreach loop is:

\begin{lstlisting}
    foreach(<type of elements in collection> <identifier> in <array or collection>)
        <statements or block of statements>
        
\end{lstlisting}

Let’s now make our previous program to iterate through the array with a foreach loop:

\begin{lstlisting}
    // demonstrates the use of arrays in C#
    static void Main()
    {
        // declaring and initializing an array of type integer
        int [] integers = {3, 7, 2, 14, 65};
        // iterating through the array and printing each element
        foreach(int i in integers)
        {
            Console.WriteLine(i);
        }
    }
\end{lstlisting}

Simple and more readable, isn’t it? In the statement:

\begin{lstlisting}
    foreach(int i in integers)    
\end{lstlisting}

We specified the type of elements in the collection (int in our case). We declared the variable (i) to be used to hold
the individual values of the array ’integers’ in each iteration.\\

Important points to note here:

\begin{itemize}
    \item The variable used to hold the individual elements of array in each iteration (i in the above example) is read
    only. You can’t change the elements in the array through it. This means that foreach will only allow you to
    iterate through the array or collection and not to change the contents of it. If you wish to perform some
    work on the array to change the individual elements, you should use a for loop.
    \item foreach can be used to iterate through arrays or collections. By a collection, we mean any class, struct or
    interface that implements the IEnumerable interface. (Just go through this point and re-read it once we
    complete the lesson describing classes and interfaces)
    \item The string class is also a collection of characters (implements IEnumerable interface and returns char
    value in Current property). The following code example demonstrates this and prints all the characters in
    the string.
     
\end{itemize}

\begin{lstlisting}
    static void Main()
    {
        string name = "Sanfy In";
        foreach(char ch in name)
        {
            Console.WriteLine(ch);
        }
    } 
\end{lstlisting}

This will print each character of the name in a separate line.
