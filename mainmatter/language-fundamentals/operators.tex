\section{Operators}
\textbf{Arithmetic Operators}\\

Several common arithmetic operators are allowed in C\#.

\begin{center}
    \begin{tabular}{ | m{5em} | m{10cm} | } 
    \hline
    Operand & Description \\
    \hline
    + & Add\\
    - & Subtract\\
    * & Multiply\\
    / & Divide\\
    \% & Remainder or modulo\\
    ++ & Increment by 1\\
    -- & Decrement by 1\\
    \hline
    \end{tabular}
\end{center}


\begin{lstlisting}
using System;
namespace Training
{
    class Program
    {
        // The program shows the use of arithmetic operators
        // + - * / % ++ --
        static void Main()
        {
            // result of addition, subtraction,
            // multiplication and modulus operator
            int sum=0, difference=0, product=0, modulo=0;
            float quotient=0; // result of division
            int num1 = 10, num2 = 2; // operand variables
            sum = num1 + num2;
            difference = num1 - num2;
            product = num1 * num2;
         
            quotient = num1 / num2;
            // remainder of 3/2
            modulo = 3 % num2;
            Console.WriteLine("num1 = {0}, num2 = {1}", num1, num2);
            Console.WriteLine();
            Console.WriteLine ("Sum of {0} and {1} is {2}", num1, num2, sum);
            Console.WriteLine("Difference of {0} and {1} is {2}", num1, num2, difference);
            Console.WriteLine("Product of {0} and {1} is {2}", num1, num2, product);
            Console.WriteLine("Quotient of {0} and {1} is {2}", num1, num2, quotient);
            Console.WriteLine();
            Console.WriteLine("Remainder when 3 is divided by {0} is {1}", num2, modulo);
            num1++; // increment num1 by 1
            num2--; // decrement num2 by 1
            Console.WriteLine("num1 = {0}, num2 = {1}", num1, num2);
        }
    }
}
\end{lstlisting}

Although the program above is quite simple, I would like to discuss some concepts here. In the
\emph{Console.WriteLine()} method, we have used format-specifiers \{int\} to indicate the position of variables in the
string.\\

\begin{lstlisting}
    Console.WriteLine("Sum of {0} and {1} is {2}", num1, num2, sum);
\end{lstlisting}


Here, \{0\}, \{1\} and \{2\} will be replaced by the values of the num1, num2 and sum variables. In {i}, i specifies that
(i+1)th variable after double quotes will replace it when printed to the Console. Hence, {0} will be replaced by the
first one, \{1\} will be replaced by the second variable and so on...

Another point to note is that num1++ has the same meaning as:

num1 = num1 + 1;

Or:

num1 += 1;\\

\textbf{Prefix and Postfix notation}\\

Both the ++ and -- operators can be used as prefix or postfix operators. In prefix form:\\

\begin{lstlisting}
    num1 = 3;
    num2 = ++num1; // num1 = 4, num2 = 4        
\end{lstlisting}

The compiler will first increment num1 by 1 and then will assign it to num2. While in postfix form:

\begin{lstlisting}
    num2 = num1++; // num1 = 4, num2 = 3    
\end{lstlisting}


The compiler will first assign num1 to num2 and then increment num1 by 1.

\textbf{Assignment Operators}\\
Assignment operators are used to assign values to variables. Common assignment operators in C\# are:

\begin{center}
    \begin{tabular}{ | m{5em} | m{10cm} | } 
    \hline
    Operand & Description \\
    \hline
    = & Simple assignment\\
    += & Additive assignment\\
    -= & Subtractive assignment\\
    *= & Multiplicative assignment\\
    /= & Division assignment\\
    \%= & Modulo assignment\\
    \hline
    \end{tabular}
\end{center}

The equals (=) operator is used to assign a value to an object. Like we have seen

\begin{lstlisting}
    bool isPaid = false;    
\end{lstlisting}

assigns the value ’false’ to the isPaid variable of Boolean type. The left hand and right hand side of the equal or any
other assignment operator must be compatible, otherwise the compiler will complain about a syntax error.
Sometimes casting is used for type conversion, e.g., to convert and store a value in a variable of type double to a
variable of type int, we need to apply an integer cast.

\begin{lstlisting}
    double doubleValue = 4.67;
    // intValue will be equal to 4
    int intValue = (int) doubleValue;
\end{lstlisting}

Of course, when casting there is always a danger of some loss of precision; in the case above, we only got the 4 of
the original 4.67. Sometimes, the casting may result in strange values:

\begin{lstlisting}
    int intValue = 32800;
    short shortValue = (short) intValue;
    // shortValue would be equal to -32736        
\end{lstlisting}

Variables of type short can only take values ranging from -32768 to 32767, so the cast above can not assign 32800
to shortValue. Hence shortValue took the last 16 bits (as a short consists of 16 bits) of the integer 32800, which
gives the value -32736 (since bit 16, which represents the value 32768 in an int, now represents -32768). If you try
to cast incompatible types like:

\begin{lstlisting}
    bool isPaid = false;
    int intValue = (int) isPaid;        
\end{lstlisting}

It won’t get compiled and the compiler will generate a syntax error.

\textbf{Relational Operators}
Relational operators are used for comparison purposes in conditional statements. Common relational operators in
C\# are:

\begin{center}
    \begin{tabular}{ | m{5em} | m{10cm} | } 
    \hline
    Operand & Description \\
    \hline
    == & Equality check\\
    != & Un-equality check\\
    > & Greater than\\
    < & Less than\\
    >= & Greater than or equal to\\
    <= & Less than or equal to\\
    \hline
    \end{tabular}
\end{center}

Relational operators always result in a Boolean statement; either true or false. For example if we have two
variables

\begin{lstlisting}
    int num1 = 5, num2 = 6;    
\end{lstlisting}

Then:

\begin{lstlisting}
    num1 == num2 // false
    num1 != num2 // true
    num1 > num2 // false
    num1 < num2 // true
    num1 <= num2 // true
    num1 >= num2 // false    
\end{lstlisting}

Only compatible data types can be compared. It is invalid to compare a bool with an int, so if you have

\begin{lstlisting}
    int i = 1;
    bool b = true;    
\end{lstlisting}

you cannot compare i and b for equality (i==b). Trying to do so will result in a syntax error.

\textbf{Logical and Bitwise Operators}\\
These operators are used for logical and bitwise calculations. Common logical and bitwise operators in C\# are:

\begin{center}
    \begin{tabular}{ | m{5em} | m{10cm} | } 
    \hline
    Operand & Description \\
    \hline
    \& & Bitwise AND\\
    | & Bitwise OR\\
    \textasciicircum & Bitwise XOR\\
    ! & Bitwise NOT\\
    \hline
    \end{tabular}
\end{center}

The operators \&, | and \textasciicircum are rarely used in usual programming practice. The NOT operator is used to negate a
Boolean or bitwise expression like:

\begin{lstlisting}
    bool b = false;
    bool bb = !b;
    // bb would be true        
\end{lstlisting}


Logical Operators \&\& and || are used to combine comparisons like

\begin{lstlisting}
    int i=6, j=12;
    bool firstVar = i>3 && j<10;
    // firstVar would be false
    bool secondVar = i>3 || j<10;
    // secondVar would be true        
\end{lstlisting}

In the first comparison: i>3 \&\& j<10 will result in true only if both the conditions i>3 and j<10 result in true.
In the second comparison: i>3 || j<10 will result in true if any of the conditions i>3 and j<10 result in true. You
can, of course, use both \&\& and || in single statement like:

\begin{lstlisting}
    bool firstVar = (i>3 && j<10) || (i<7 && j>10) // firstVar would be true    
\end{lstlisting}


In the above statement we used parenthesis to group our conditional expressions and to avoid any ambiguity.
You can use \& and | operators in place of \&\& and || but for combining conditional expressions, \&\& and || are more
efficient because they use "short circuit evaluation". For example, if in the expression (i>3 \&\& j<10), i>3
evaluates to false, the second expression j<10 won’t be checked and false will be returned (when using AND, if one
of the participant operands is false, the whole operation will result in false). Hence, one should be very careful
when using assignment expressions with \&\& and || operators. The \& and | operators don’t do short circuit
evaluation and do execute all the comparisons before returning the result.\\


\textbf{Other Operators}\\
There are some other operators present in C\#. A short description of these is given below:

\begin{center}
    \begin{tabular}{ | m{5em} | m{10cm} | } 
    \hline
    Operand & Description \\
    \hline
    << & Left shift bitwise operator\\
    >> & Right shift bitwise operator\\
    . & Member access for objects\\
    "" & Index operator used in arrays and collections\\
    () & Cast operator\\
    ?: & Ternary operator\\
    \hline
    \end{tabular}
\end{center}

\textbf{Operator Precedence}\\
All operators are not treated equally. There is a concept of "operator precedence" in C\#. For example:

\begin{lstlisting}
    int i = 2 + 3 * 6;
    // i would be 20 not 30        
\end{lstlisting}
3 will be multiplied by 6 first then the result will be added to 2. This is because the multiplication operator * has
precedence over the addition operator +. For a complete table of operator precedence, consult MSDN or the .Net
framework documentation.