\chapter{Introduction}

C\# (pronounced C-Sharp) is an Object Oriented Programming language and it has many similarities to Java, C++ and VB. In fact, C\#
combines the power and efficiency of C++, the simple and clean OO design of Java and the language simplification of Visual Basic.\\

\textbf{Writing Your First Hello World Program in C\#}

We can use any Text Editor for writting a C\# program. We will use Visual Studio Code for this training.\\

Create a file as Program.cs, write below Code and save the file.

\begin{lstlisting}
class Program
{
    public static void Main(string[] args)
    {
        System.Console.WriteLine("Hello World!");
    }
}
\end{lstlisting}

To compile this file, go to command prompt (Windows) or Terminal (Mac) and execute below command:


\colorbox{lightgray}{csc Program.cs}


This will compile your program and create an .exe file (Program.exe) in the same directory and will
report any errors that may occur.\\

To run your program, in the command prompt or terminal, type \colorbox{lightgray}{Program} in Windows or \colorbox{lightgray}{mono Program} in Mac.\\

This will print Hello World as a result on your console screen. Simple, isn’t it?\\


\textbf{Understanding the Hello World Application Code:}


\textbf{The class Keyword}\\
All of our C\# programs contain at least one class. The Main() method resides in one of these classes. Classes are a
combination of data (fields) and functions (methods) that can be performed on this data in order to achieve the
solution to our problem. We will see the concept of class in more detail later. Classes in C\# are
defined using the class keyword followed by the name of class.\\

\textbf{The Main() Method}\\
In the next line we defined the Main() method of our program:\\

\colorbox{lightgray}{public static void Main(string[] args)}\\

This is the standard signature of the Main method in C\#. The Main method is the entry point of our program, i.e.,
our C\# program starts its execution from the first line of Main method and terminates with the termination of Main
method.\\

\textbf{Printing on the Console}\\
Our next line prints Hello World on the Console screen:\\

\colorbox{lightgray}{Console.WriteLine("Hello World");}\\

Here we called WriteLine(), a static method of the Console class defined in the System namespace. This method
takes a string (enclosed in double quotes) as its parameter and prints it on the Console window.
C\#, like other Object Oriented languages, uses the dot (.) operator to access the member variables (fields) and
methods of a class. Also, braces () are used to identify methods in the code and string literals are enclosed in double
quotation marks ("). Lastly, each statement in C\# (like C, C++ and Java) ends with a semicolon (;), also called the
statement terminator.


\textbf{Namespaces in C\#}\\
A Namespace is simply a logical collection of related classes in C\#. We bundle our related classes in some named collection calling it a namespace. As C\# does
not allow two classes with the same name to be used in a program, the sole purpose of using namespaces is to
prevent name conflicts. This may happen when you have a large number of classes, as is the case in the Framework
Class Library (FCL). It is very much possible that our Connection Class in DataActivity conflicts with the
Connection Class of InternetActivity. To avoid this, these classes are made part of their respective namespace. So
the fully qualified name of these classes will be DataActivity.Connection and InternetActivity.Connection, hence
resolving any ambiguity for the compiler.

We can define the namspace \emph{Training} in our program as below:

\begin{lstlisting}

namespace Training
{
    class Program
    {
        public static void Main()
        {
            System.Console.WriteLine("Hello World!");
        }
    }
}
\end{lstlisting}

Now we can access Program class and Main method from anywhere in a program as below:

\begin{lstlisting}
    Training.Program.Main();    
\end{lstlisting}


\textbf{The using Keyword}\\

We can write the namespaces in a using statement at the start of a program as below:

\begin{lstlisting}
    using System;

    namespace Training
    {
        class Program
        {
            public static void Main()
            {
                Console.WriteLine("Hello World!");
            }
        }
    }
\end{lstlisting}

The \emph{using} keyword above allows us to use the classes in the ’System’ namespace. By doing this, we can
now access all the classes defined in the System namespace like we are able to access the Console class in our
Main method.
		
